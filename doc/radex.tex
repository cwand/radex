\documentclass[a4paper,article]{memoir}

% Pakker, der skal bruges sideopsætning, billeder m.m.
\usepackage[latin1]{inputenc}
\usepackage[danish]{babel}
%\renewcommand\danishhyphenmins{22}
\usepackage[T1]{fontenc}
\usepackage[margin=3cm]{geometry}
\usepackage{graphicx}
\usepackage{xspace}
\usepackage{color}
\usepackage{array,booktabs}
\usepackage{url}
\usepackage{csquotes}

% Matematik og symboler
\usepackage{amsmath,amssymb}
\usepackage{bm}
\usepackage{amsthm}
\usepackage{mathtools}
\usepackage{listings}
\usepackage{verbatim}

% Fysik pakker
\usepackage{physics}
\usepackage{siunitx}
\usepackage{isotope}

% Pakke der kan lave rette-noter i margin
\usepackage[draft,danish]{fixme}

\newcommand{\radex}{\emph{Radex}}

% ====================================================================
% HUSK AT ÆNDRE TITEL OG FORFATTER
\geometry{headheight=1cm}
\title{Radex} 	% Dokumentes Titel
\author{Christian Walther Andersen}		% Dokumentets forfatter
%%% \date{i dag}		% Her kan datoen indsættes, men udelades kommandoen indsætter den altid dags dato
% ====================================================================

\begin{document}
	
\setlength{\parindent}{0pt}

\maketitle
\fancybreak{$*\quad*\quad*$}
\vspace{5mm}

% Dokumentet begynder her

\section*{Brug af programmet}

\radex{} anvendes til at detektere tilstedev�relsen af isotopen 
\isotope[223]{Ra}.
Det tilt�nkte brugsscenarie er, at en m�ngde af affald, t�j e.l., der har v�ret
i n�rkontakt med \isotope[223]{Ra}, skal sikres, inden det smides ud, 
genanvendes e.l. Affaldet scannes med et gamma-kamera, og billedet, 
der ogs� indeholder det m�lte spektrum i dicom-format, analyseres for spor af 
\isotope[223]{Ra}. Det er denne sidste opgave, der kan klares af \radex{}.


\section*{Download og installation}

Kildekoden kan downloades fra GitHub: \url{https://github.com/cwand/radex}. 
\radex{} er skrevet i Python og kan bruges uden installation. Der er dog et par 
variable, der skal angives i kildekoden, inden programmet er helt klar til at 
k�re.

\subsection*{Indstilling af filstier}

\radex{} kigger efter dicom-billeder i en filsti, der er angivet i filen 
\texttt{main.py}. Filstierne er angivet t�t p� toppen af filen. Variablen 
\texttt{pardir} angiver hvor \radex{} skal kigge efter dicom-filer. \radex{} 
leder i undermapper efter alle filer med filendelsen \texttt{.dcm}. Ogs� 
baggrundsm�linger skal ligge i denne filsti.

Variablen \texttt{archdir} angiver, hvor \radex{} l�gger dicom-filerne, n�r den 
er f�rdig med analysen. N�r analysen er f�rdig flyttes indholdet af 
\texttt{pardir}-mappen til \texttt{archdir} (man kan v�lge dette fra i 
slutningen af analysen, hvis dette ikke �nskes).

\subsection*{Installation}

Hvis man �nsker at installere \radex{}, fx hvis programmet skal bruges p� en 
arbejdsstation, der ikke har Python installeret, kan dette g�res ved brug af 
Python-modulet \texttt{pyinstaller}:

\begin{verbatim}
	>>> pyinstaller main.spec
\end{verbatim}

N�r installationen er fuldf�rt findes en ny mappe med navnet \texttt{dist}, der 
indeholder alt, der er n�dvendigt for at k�re programmet. Denne mappe kan s� 
kopieres over p� den tilt�nkte arbejdsstation, og programmet kan k�res derfra.

\end{document}
